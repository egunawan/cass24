%%%%%%%%%%%%%%%%%%%%%%%%%%%%%%%%%%%%%%%%%%%%%%%%%%%%%%%
% Math 3250 Combinatorics, University of Connecticut
%%%%%%%%%%%%%%%%%%%%%%%%%%%%%%%%%%%%%%%%%%%%%%%%%%%%%%%

% Anything after a percent sign is a comment.

% Necessary first line. \documentclass defines the type of document and some options (for example, try changing the font size (10pt or 11pt or 12pt).
\documentclass[10pt]{amsart}
\usepackage{enumerate} % to be able to enumerate a list using non-numbers
\usepackage{tikz}
%for hypertext references
\usepackage[colorlinks = true,
            linkcolor = blue,
            urlcolor  = blue,
            citecolor = red,
            anchorcolor = green]{hyperref}


\usepackage[letterpaper]{geometry} 
\geometry{tmargin=0.53in,bmargin=-0.0in,lmargin=1.00in,rmargin=1.00in}
\voffset -0.5in

% The part of the tex file between the \documentclass and \begin{document} line is called the preamble.  Things having to do with setting up the document are done in the preamble.  You will not need to mess with it for now, except to change your name and the title of your document. 
% After adding your name next to author, head down to where it says "Start here"

\title{Math3250 Combinatorics Reading HW 9}
%\author{your preferred first and last name:}
%\date{}
\begin{document}

\maketitle

% --------------------------------------------------------
%                         Start here
% --------------------------------------------------------

%\noindent Credit: 
%Write down everyone who helped you, including classmates who contributed to your thought process. Write down Bona's textbook and other written sources you used as well.

\subsection*{Instruction}
Please submit all sections. 
Because you will be doing a lot of arithmetic and sketching, it might be more convenient to do this homework by hand.
Use B\'ona's ``A Walk through Combinatorics" textbook.

\bigskip


\section*{Review of Sec 8.1.2 The Product Formula}
(If you understood the lecture during class on Thursday, March 5, then this part is optional)
Write down the problem and solution to Example 8.8 page 172. Explain the computation details skipped by the textbook.


\section{Lucas and other Fibonacci/Pingala-like numbers}
(This is another example related to 	Section 8.1.1: 
	Recurrence relations and generating functions)
	
	
Let $a_1=1$, $a_2=3$, and $a_n=a_{n-1} + a_{n-2}$ for all $n \geq 1$.



\begin{enumerate}
	\item 
	Compute a simple explicit formula for an ordinary generating function of $a_n$.
	\item 
	Compute a closed-form (no summation) formula for $a_n$.
\end{enumerate}

	
Note: If you are not sure how to do the problems, you can follow the first ten minutes of this Jim Fowler's video for computing a formula for the Fibonacci/Pingala numbers \href{https://www.youtube.com/watch?v=CR-nmp97Ayo}{youtube.com/watch?v=CR-nmp97Ayo}. Your final answer will be slightly different because your initial values are different, but you can follow his methods exactly.




\section{Catalan numbers from the Internet}
\begin{enumerate}[a.]
	\item Pick two types of Catalan objects. For each type, draw the fourteen objects for $n=4$ (so you will draw a total of 28 pictures). 
	You can pick the ones introduced during class, or you can pick two objects from this Catalan numbers page \href{http://mathshistory.st-andrews.ac.uk/Miscellaneous/CatalanNumbers/catalan.html}{http://mathshistory.st-andrews.ac.uk/Miscellaneous/CatalanNumbers/catalan.html}. The Wikipedia entry for ``Catalan number" also has a lot of nice pictures of some Catalan objects. 	
	\item 
	Spend about fifteen minutes reading a few pages linked from this Catalan numbers page,  by Igor Pak:\\
	 \href{https://www.math.ucla.edu/~pak/lectures/Cat/pakcat.htm}{https://www.math.ucla.edu/~pak/lectures/Cat/pakcat.htm}. Then 
	write down two new facts about the Catalan numbers that you learned from these pages. 
	\item Attempt to find a bijection between two Catalan objects. If you can't, then read and explain one of the bijections given in these slides by Richard Stanley (just do ctrl+F or Apple-F for the word "bijection"):
\href{https://math.mit.edu/~rstan/transparencies/china.pdf}{https://math.mit.edu/~rstan/transparencies/china.pdf}

\end{enumerate}



\section{Catalan numbers: Tuples}
We say that a tuple $(z_0=1, z_1, z_2, \dots, z_{n-1}, z_n=1)$ of positive integers  is \emph{admissible} 
if, for all $k \in \{1, \dots, n-1\}$, the integer $x_k$ divides $(x_{k-1} + x_{k+1})$. 
For example, there are exactly five admissible tuples for $n=3$,
\begin{center}
$(1,1,1,1)$, $(1,1,2,1)$, $(1,2,1,1)$, $(1,2,3,1)$, and $(1,3,2,1)$.
\end{center}

\begin{enumerate}[a.]
\item 
For fifteen minutes, write down as many of the fourteen admissible tuples as you can for $n=4$. It's OK if you don't find all fourteen.
		
\item 
Attempt to find a bijection between these admissible tuples and other Catalan objects you wrote about earlier (triangulations, trees, etc). Try googling. It's OK if you don't succeed.

\end{enumerate}




\section{Catalan numbers Sec 8.1.2.1}

\begin{enumerate}[i.]
\item Go to Sec 8.1.2.1
Catalan numbers (pg 174--176). Read the pages a few times, until they make sense.

\item 
Write down your notes (from reading the pages). Include the computation needed to get to Equation (8.14) and the formula $c_n=\frac{\binom{2n}{n}}{n+1}$ which were skipped by the book.

\end{enumerate}





\section{Survey}
\begin{enumerate}[i.] 
\item Approximately how much time did you spend on this homework?
%\item Write down the resources (for example, B\'ona's textbook or a  \href{https://math.stackexchange.com/}{math.stackexchange.com/} page) you referenced and the people that you talked with.	
\item Questions or comments?
\end{enumerate}



\end{document}
