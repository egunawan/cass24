%%%%%%%%%%%%%%%%%%%%%%%%%%%%%%%%%%%%%%%%%%%%%%%%%%%%%%%
% Math 3250 Combinatorics, University of Connecticut
%%%%%%%%%%%%%%%%%%%%%%%%%%%%%%%%%%%%%%%%%%%%%%%%%%%%%%%

% Anything after a percent sign is a comment.

% Necessary first line. \documentclass defines the type of document and some options (for example, try changing 12pt to 10pt.
\documentclass[12pt]{amsart}


% The part of the tex file between the \documentclass and \begin{document} line is called the preamble.  Things having to do with setting up the document are done in the preamble.  You will not need to mess with it for now, except to change your name and the title of your document. 
% After adding your name next to author, head down to where it says "Start here"

\title{Math3250 Combinatorics Sample Problem Set}
\author{Mikl\'os B\'ona (typed by Emily Gunawan)}
\date{}
\begin{document}

\maketitle

% --------------------------------------------------------
%                         Start here
% --------------------------------------------------------


Note: In this class, we will use $[n]$ to denote the set $\{1,2,\dots,n \}$.

\section{The number of all subsets (Theorem 2.4 page 27)}
For all positive integers $n$, the number of all subsets of $[n]$ is $2^n$.

\begin{proof}[Proof (by induction)]
For $n=1$, the statement is true as $[1]=\{1\}$ has two subsets, the empty set, and $\{1\}$.

Now let $k$ be a positive integer, and assume that the statement is true for $n=k$. We divide the subset of $[k+1]$ into two classes: there will be those subsets that do not contain the element $k+1$, and there will be those that do. Those that do not contain $k+1$ are also subsets of $[k]$, so by the induction hypothesis their number is $2^k$. Those that contain $k+1$ consist of $k+1$ and a subset of $[k]$. However, that subset of $[k]$ can be any of the $2^k$ subsets of $[k]$, so the number of these subsets of $[k+1]$ is once more $2^k$. So altogether, $[k+1]$ has $2^k + 2^k = 2^{k+1}$ subsets, and the statement is proven. 
\end{proof}

\end{document}
