%%%%%%%%%%%%%%%%%%%%%%%%%%%%%%%%%%%%%%%%%%%%%%%%%%%%%%%
% Math 3250 Combinatorics, University of Connecticut
%%%%%%%%%%%%%%%%%%%%%%%%%%%%%%%%%%%%%%%%%%%%%%%%%%%%%%%

% Anything after a percent sign is a comment.

% Necessary first line. \documentclass defines the type of document and some options (for example, try changing 12pt to 10pt.
\documentclass[12pt]{amsart}
\usepackage{enumerate} % to be able to enumerate a list using non-numbers
\usepackage{tikz}
%for hypertext references
\usepackage[colorlinks = true,
            linkcolor = blue,
            urlcolor  = blue,
            citecolor = red,
            anchorcolor = green]{hyperref}


% The part of the tex file between the \documentclass and \begin{document} line is called the preamble.  Things having to do with setting up the document are done in the preamble.  You will not need to mess with it for now, except to change your name and the title of your document. 
% After adding your name next to author, head down to where it says "Start here"

\title{MT3250 Combinatorics Week10 Problem Set}
\author{your preferred first and last name}
\date{}
\begin{document}

\maketitle

% --------------------------------------------------------
%                         Start here
% --------------------------------------------------------

\noindent Credit: 
Write down everyone who contributed to your thought process. Write down Bona's textbook and other written sources you used as well.

\subsection*{Instruction (please comment out or erase)} 
\begin{itemize}
\item %Please send me an invite via Overleaf. 
You are encouraged to work with other people, but this week your must write your own solution. 

\item 
Because the Calculus 2 computation may be tedious to type, you can submit handwritten work at the beginning of class (or type your work, if you prefer). 
If you have to take partial fractions or a complicated derivative or antiderivative, 
you may use a computing tool (or not), but do the rest of your work by hand. 

\item You should check your computation with a computing tool like WolframAlpha or Mathematics, etc (free for UConn students). 
\end{itemize}

\bigskip 


\section{%Section 8.1.1: 
Recurrence relations and generating functions}
Let $a_1=1$, $a_2=3$, and $a_n=a_{n-1} + a_{n-2}$ for all $n \geq 1$.\footnote{Compute $a_3$, $a_4$, and $a_5$ and observe that this is not the Fibonacci sequence.}



\begin{enumerate}
\item 
Compute a simple explicit formula for a generating function (either ordinary or exponential) of $a_n$.
\item 
Compute a closed-form (no summation) formula for $a_n$.

{\tiny \noindent Uncomment for a hint:}
%\\ Since a recurrence relation is given and the sequence is not growing rapidly, you can try using the \emph{ordinary} generating function technique used in the examples of Section 8.1.1 Recurrence Relation and \emph{Ordinary} Generating Function. \begin{enumerate} [i.]\item Compute the OGF. \item Use the quadratic formula to find two (irrational) roots. \item Apply partial fraction. \item One way to compute the formula for $a_n$ is similar to Example 8.14 page 178. \item Watch this Jim Fowler's video for computing a formula for the Fibonacci numbers \href{https://www.youtube.com/watch?v=CR-nmp97Ayo}{youtube.com/watch?v=CR-nmp97Ayo} \end{enumerate}

\end{enumerate}

\begin{proof}[Answer]
An explicit formula for the (pick one: ordinary/exponential) generating function is \[\boxed{insert ~ formula ~ in ~ terms ~ of ~ x}.\]

An explicit formula for $a_n$ is $\boxed{insert ~ formula ~ in ~ terms ~ of ~ n}$.
\end{proof}




\section{
Binary trees
}
 
A \emph{binary tree} is a rooted tree where each vertex has a left subtree and a right subtree (which may be empty). In other words, a binary tree is a rooted tree where each vertex has at most 2 children.\footnote{Note that this definition is different from the \emph{full} binary trees described during lecture.} 
See slide 18 \url{http://www-math.mit.edu/~rstan/transparencies/china.pdf} for the 5 binary trees with $n=3$ vertices. %Several other Catalan objects are explained there as well.

Let $b_0=1$, and let $b_n$ be the number of binary trees with $n$ vertices for $n \geq 1$. Here are the first few values of $b_n$.
\begin{center}
\begin{tabular}{ l c }
  $n$ & $b_n$  \\
  \hline
   0 & 1 \\
  1 & 1 \\
  2 & 2  \\
  3 & 5 \\
  4 & 14 \\
  5 & 42
\end{tabular}
\end{center}
\begin{enumerate}
\item Prove that \[b_n=b_0 b_{n-1} + b_1 b_{n-2} + b_2 b_{n-3} + \dots + b_{n-1} b_0.\]

{\tiny \noindent Uncomment for hints:}
%You should follow the proof from lecture (for polygon triangulations) or Bona's proof for Chapter 8 Exercise 16 or find a proof on the internet or another textbook. 

\item Compute the ordinary generating function $B(x)=\sum_{n=1}^\infty b_n x^n$ for $b_n$. 

{\tiny \noindent Uncomment for hints:}
%You can either use the product formula like in Section 8.1.2.1 of Bona or use the recurrence relation like in the lecture's proof (for generating functions for polygon triangulations). 

 \end{enumerate}


\section{%Section 8.2.1
Permutations
}

Let $d_0=1$ and, if $n \geq 1$, let $d_n$ be the number of bijections on $[n]$ such that, for all $i\in [n]$, it sends $i$ to another number. 
Then $d_n$ satisfies the recurrence (you are not asked to prove this) 
\begin{equation}
\label{eq:dn}
d_n = n ~ d_{n-1} + (-1)^n.
\end{equation}

\begin{enumerate}[a.]
\item
Use (\ref{eq:dn}) to compute the \emph{exponential} generating function $D(x)=\sum_{n=1}^\infty d_n \frac{x^n}{n!}$ for $d_n$. (Hint: Follow the examples in Sec 8.2.1 Recurrence Relations and Exponential Generating Functions).

\item
Use it to prove that  
\[
d_n = n! \sum_{j=0}^n \frac{(-1)^j}{j!}
\]

{\tiny  \noindent Uncomment for a hint:}
% \begin{enumerate}[i.] \item You've shown that $D(x)$ is equal to the product of two function of $x$. \item Write this as a product of two power series. \item There are no binomial coefficients in either series, so use Lemma 8.4 for \emph{ordinary} generating function to write $D(x)$ as one power series. Show work. \end{enumerate}



\end{enumerate}

\section{%Section 8.2.1: 
Harmonic numbers}

\begin{enumerate}[a.]
\item
Compute by hand the Taylor series for $\ln(\frac{1}{1-x})$ centered at $0$. 
Read your Calculus textbook or watch lecture videos explaining this specific problem.

{\tiny \noindent Uncomment for a hint: }
% \begin{enumerate}[i.] \item Write out the Taylor series for $\frac{1}{1-x}$ \item Integrate both sides. Don't forget to compute what the constant ``C" has to be. \item Check your answer with a computing tool. \end{enumerate}


\item
Let $h_1=0$  and \[h_{n+1}= (n+1) ~h_{n} + n! ~~ \text{ for all $n \geq 0$}. \]
Compute an explicit formula for a generating function (either ordinary or exponential) of $h_n$.
Use it to prove the formula 
\begin{equation*}
h_n = n! \sum_{k=1}^n \frac{1}{k}.
\end{equation*}

{\tiny \noindent Uncomment for a hint: }
%\\ Since a recurrence relation is given and because of the factorial in the recursive formula, you can try using the \emph{exponential} generating function technique used in the examples of Section 8.2.1 Recurrence Relation and \emph{Exponential} Generating Function. \begin{enumerate}[i.] \item Multiply both sides of the recurrence relation by $x^{n+1}/(n+1)!$ and sum over all $n \geq 0$. \item Let $H(x)$ be the exponential generating function for $h_n$. Show that $H(x)= \frac{1}{1-x} \ln\left(\frac{1}{1-x}\right)$. \item Write this as a product of two power series. \item There are no binomial coefficients in either series, so use Lemma 8.4 for \emph{ordinary} generating function to write $H(x)$ as one power series. Show work. \item The ``extra" $n!$ factor is because $H(x)=\sum_{n=1}^\infty h_n \frac{x^n}{n!}$. \end{enumerate}

\end{enumerate}

\begin{proof}[Answer]
\begin{enumerate}
\item The Taylor series for $\ln\left(\frac{1}{1-x}\right)$  is \[\boxed{ \sum_{n=}^\infty ~ insert ~ x^n}\]

\item 
An explicit formula for the (pick one: ordinary/exponential) generating function is \[\boxed{insert ~ formula ~ in ~ terms ~ of ~ x}.\]

An explicit formula for $a_n$ is $\boxed{insert ~ formula ~ in ~ terms ~ of ~ n}$.

\end{enumerate}
\end{proof}






\section{Section 8.2.2: The product formula}
Given a classroom with $n$ students ($n \geq 0$), 
a teacher divides the students into three groups\footnote{groups can be empty} $A$, $B$, $C$ so that $A$ has an odd number of people and $B$ has an even number of people (no restriction for $C$). The teacher then asks each group to form a line. 

Let $f_n$ be the number of ways to do this. 
Find a generating (either ordinary or exponential) function $F(x) = \sum_{n=0}^\infty f_n ~ x^n$ and use it to find a closed form formula for $f_n$. 

{\tiny \noindent Uncomment for a hint:}
%\\This is very similar to the ``Section 8.1.2: The product formula" problem from the previous problem set. Define three sequences, the number of ways line up $n$ people (if $n$ is odd), the number of ways to line up $n$ people (if $n$ is even), and the number of ways to line up $n$ people. We have to use the \emph{exponential} (as opposed to ordinary) generating function product formula because a group is not an interval (unlike an interval of  dollars to be paid with only, say, $5$ dollar bills). Follow the three examples given in Section 8.2.2. Use partial fractions to find a closed form formula for $f_n$. 




\section{Miscellaneous}
\begin{enumerate}[i.]
    \item Share your work (at least one problem) and thought process with at least one classmate. Ask them to share their thought process as well. Write down their names and briefly summarize your interactions. A virtual discussion via Piazza or email is fine if you don't have time to interact in person. 
    \item Approximately how much time did you spend on this homework?
\end{enumerate}

\end{document}
