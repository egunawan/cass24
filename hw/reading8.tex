%%%%%%%%%%%%%%%%%%%%%%%%%%%%%%%%%%%%%%%%%%%%%%%%%%%%%%%
% Math 3250 Combinatorics, University of Connecticut
%%%%%%%%%%%%%%%%%%%%%%%%%%%%%%%%%%%%%%%%%%%%%%%%%%%%%%%

% Anything after a percent sign is a comment.

% Necessary first line. \documentclass defines the type of document and some options (for example, try changing the font size (10pt or 11pt or 12pt).
\documentclass[10pt]{amsart}
\usepackage{enumerate} % to be able to enumerate a list using non-numbers
\usepackage{tikz}
%for hypertext references
\usepackage[colorlinks = true,
            linkcolor = blue,
            urlcolor  = blue,
            citecolor = red,
            anchorcolor = green]{hyperref}


\usepackage[letterpaper]{geometry} % from 2016
\geometry{tmargin=0.53in,bmargin=-0.0in,lmargin=1.00in,rmargin=1.00in}
\voffset -0.5in

% The part of the tex file between the \documentclass and \begin{document} line is called the preamble.  Things having to do with setting up the document are done in the preamble.  You will not need to mess with it for now, except to change your name and the title of your document. 
% After adding your name next to author, head down to where it says "Start here"

\title{Math3250 Combinatorics Reading HW 8}
%\author{your preferred first and last name:}
%\date{}
\begin{document}

\maketitle

% --------------------------------------------------------
%                         Start here
% --------------------------------------------------------

%\noindent Credit: 
%Write down everyone who helped you, including classmates who contributed to your thought process. Write down Bona's textbook and other written sources you used as well.

\subsection*{Instruction}
Please submit all sections. 
Because you will be doing a lot of arithmetic, it might be more convenient to do this homework by hand.



Go to Sec 8.1.1 and Sec 8.1.2 of B\'ona's ``A Walk through Combinatorics" textbook, pg. 163--172.

\bigskip


\section*{Review}
Optional: Read pages 163--167, including Example 8.1 (given as lectures on Tuesday, Feb 25).


\section{Example 8.2}
\begin{enumerate}[a.]
	\item 
Attempt the problem of Example 8.2 (page 167) using generating function method without reading the solution. \\

Note:
You can use software to compute the partial fraction (see part 4 of the solution). On WolframAlpha, you can type 
\begin{center}
\texttt{partial fraction 500x/(1-x)(1-1.05x)}.
\end{center}

\item 
Check the solution in the textbook, and write the correct solution. Include details skipped by the textbook.

\item Questions and comments?
\end{enumerate}



\section{Example 8.3}
\begin{enumerate}[a.]
	\item 
	Attempt the problem of Example 8.3 (page 168) using generating function method without reading the solution. \\
	
	Note:
	You can use software to compute the partial fraction (see part 4 of the solution). On WolframAlpha, you can type 
	\begin{center}
		\texttt{partial fraction x/(x-1)(2x-1)}.
	\end{center}
	
	\item 
	Check the solution in the textbook, and write the correct solution. Include  details skipped by the textbook.
	
	\item Questions and comments?
\end{enumerate}




\section{Lemma 8.4 Coefficients of the product of two power series}

\begin{enumerate}[i.]
\item Read Lemma 8.4 several times, until you understand the statement. Then write down the statement of the lemma.

\item 
Write down the proof of Lemma 8.4
\end{enumerate}








\section{Theorem 8.5 The Product Formula}

\begin{enumerate}[i.]
	\item Read Theorem 8.5 many times, until you understand the statement. Then write down the statement of the theorem.\\
	
	Read the statement of the theorem  several more times.
	
	
	\item 
	Write down the proof of Theorem 8.5.
\end{enumerate}



\section{Example 8.6, 8.7}

\begin{enumerate}[i.]
	\item Read the problem and solution of Example 8.6 or 8.7 many times, until you understand at least part of the solution. Then write down the problem given in Example 8.6.\\
	
	
	\item 
	Write down the solution of Example 8.6 given in the book.
	
	\item Comments and questions?
\end{enumerate}


%\section{Example 8.7}
%
%\begin{enumerate}[i.]
%	\item Read the problem and solution of Example 8.7 many times, until you understand at least part of the solution. Then write down the problem given in Example 8.7.\\
%	
%	
%	\item 
%	Write down the solution of Example 8.7 given in the book.
%	
%	\item Comments and questions?
%\end{enumerate}



\section{Survey}
\begin{enumerate}[i.] 
\item Approximately how much time did you spend on this homework?
%\item Write down the resources (for example, B\'ona's textbook or a  \href{https://math.stackexchange.com/}{math.stackexchange.com/} page) you referenced and the people that you talked with.	
\item Questions or comments about this homework?
\end{enumerate}



\end{document}
