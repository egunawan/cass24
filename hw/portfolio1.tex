%%%%%%%%%%%%%%%%%%%%%%%%%%%%%%%%%%%%%%%%%%%%%%%%%%%%%%%
% Math 3250 Combinatorics, University of Connecticut
%%%%%%%%%%%%%%%%%%%%%%%%%%%%%%%%%%%%%%%%%%%%%%%%%%%%%%%

% Anything after a percent sign is a comment.

% Necessary first line. \documentclass defines the type of document and some options (for example, try changing the font size (10pt or 11pt or 12pt).
\documentclass[11pt]{amsart}
\usepackage{enumerate} % to be able to enumerate a list using non-numbers
\usepackage{tikz}
%for hypertext references
\usepackage[colorlinks = true,
            linkcolor = blue,
            urlcolor  = blue,
            citecolor = red,
            anchorcolor = green]{hyperref}


\usepackage[letterpaper]{geometry} % from 2016
\geometry{tmargin=0.93in,bmargin=-0.0in,lmargin=1.50in,rmargin=1.50in}
\voffset -0.5in

% The part of the tex file between the \documentclass and \begin{document} line is called the preamble.  Things having to do with setting up the document are done in the preamble.  You will not need to mess with it for now, except to change your name and the title of your document. 
% After adding your name next to author, head down to where it says "Start here"

\title{Math3250 Portfolio Problems 1}
%\author{your preferred first and last name:}
%\date{due at the end of class on d2, Week 1}
\begin{document}



\section*{Cover Sheet (Math3250 Combinatorics)}
Your Name: NAME

\vfill 

Assignment: Portfolio Problems 1

\vfill 

\bigskip
In this class, you are encouraged to discuss the problems with anyone. You are also allowed to use any resources as long as you cite them.

Please cite the individuals and documents that have helped you in completing this assignment. 
The individuals cited should include all classmates with whom you have collaborated (even if they only act as a sounding board by listening to you during class group work).
Documents cited should include B\'ona's textbook and all web content you have consulted.
\\


Individuals:

\vfill
\vfill

Documents:

\vfill
\vfill

\newpage


\maketitle

% --------------------------------------------------------
%                         Start here
% --------------------------------------------------------

%\noindent Credit: 
%Write down everyone who helped you, including classmates who contributed to your thought process. Write down Bona's textbook and other written sources you used as well.

\subsection*{Overleaf Setup}
In your Overleaf project ``Combinatorics portfolio by Your Name," create a new folder by clicking on the icon (shaped like a folder) on the top left. Name this folder ``portfolio1". 

To upload this template file, you can first download it to your local machine and then upload it (by clicking the upload button on the top left).
Alternatively, you can create a new blank file (click on the button shaped like paper on the top left), name it ``main.tex", then copy and paste the content of this template file to ``main.tex".

\subsection*{Instruction} 

Problem 
\ref{sec:recurrence_relation} must be done in Overleaf. 
If you are new to Overleaf, you may choose to complete the rest of the problems by hand.



Reference the  \href{https://www.worldscientific.com/doi/pdf/10.1142/9789813148857_0001}{first chapter} and second chapter of  B\'ona's ``A Walk Through Combinatorics" textbook (4th, 3rd, or 2nd Ed.) for Pigeon-Hole Principle and induction methods.

\bigskip




\section{Chapter 1 Problems}

Submit at least two problems from this section.



\subsection{Polynomial degrees}\label{sec:polynomial_degrees}
The product of five given polynomials is a polynomial of degree $21$. Prove that we can choose two of those polynomials so that the degree of their product is at least nine.







\subsection{Soccer Team} 
A soccer team scored a total of 40 goals this season. Nine players scored at least one of those goals. Prove that there are two players among those nine who scored the same number of goals.
\begin{proof}
\end{proof}



\subsection{Faculty members}
A college has $39$ departments, and a total of $262$ faculty members in those departments. Prove that there are three departments in this college that have a total of at least $21$ faculty members.
\begin{proof}
\end{proof}









\section*{Sample Solution: The number of all subsets (Thm 2.4 pg 27)}

Note: In this class, we will use $[n]$ to denote the set $\{1,2,\dots,n \}$.
Please read the source code of the following sample solution. Do you have any questions?


\smallskip


\noindent 
Prove that, for all positive integers $n$, the number of all subsets of $[n]$ is $2^n$.

\begin{proof}[Proof (by induction)]
For $n=1$, the statement is true as $[1]=\{1\}$ has two subsets, the empty set, and $\{1\}$.
	
Now let $k$ be a positive integer, and assume that the statement is true for $n=k$. We divide the subset of $[k+1]$ into two classes: there will be those subsets that do not contain the element $k+1$, and there will be those that do. Those that do not contain $k+1$ are also subsets of $[k]$, so by the induction hypothesis their number is $2^k$. Those that contain $k+1$ consist of $k+1$ and a subset of $[k]$. However, that subset of $[k]$ can be any of the $2^k$ subsets of $[k]$, so the number of these subsets of $[k+1]$ is once more $2^k$. So altogether, $[k+1]$ has $2^k + 2^k = 2^{k+1}$ subsets, and the statement is proven. 
\end{proof}


\section{polygon}
Prove (using induction) that for all $n \ge 3$ the sum of the angles of a convex $n$-gon is $(n-2)180$ degrees.

Note: You may use the fact that the sum of angles of a convex triangle is $180^o$ without proof.
\begin{proof}
	Insert proof
\end{proof}




\section{Recurrence Relation}

\subsection{Sample Solution}
Let the sequence $\{ a_n\}$ be defined by the relations $a_0=1$, and let
\[
a_{n+1} = 2(a_0 + a_1 + \dots + a_n) 
\]
for $n \geq 0$. 
Compute the first few values of $a_n$, then conjecture an explicit formula for $a_n$, and then prove the formula using induction.
\begin{proof}[Solution and proof]
	We claim that $a_n = 2 \cdot 3^{n-1}$ for $n \geq 1$. We prove this by strong induction on $n$.
	Since $2(a_0)=2(1)=2\cdot 3^{1-1}$, the initial case (for $n=1$) is verified.  
	Now let us assume that the statement is true for all positive integers that are less than or equal to $n$. Then, we have
	\begin{align*}
		a_{n+1} &= 2(a_0 + a_1 + a_2 + \dots + a_n) ~ \text{ by the recurrence relation}\\
		&= 2 a_0 + 2 (a_1 + a_2 + \dots +  a_n)\\
		&= 2 + 2(2 \cdot 1 + 2 \cdot 3 + \dots + 2 \cdot 3^{n-1}) ~ \text{ by the induction hypothesis}\\
		&= 2 + 4 ( 1 +  3 + \dots +  3^{n-1})\\
		&= 2 + 4 \left( \frac{3^n - 1}{2}  \right) ~ \text{ since the series is a geometric series} \\
		&= 2 + 2 (3^n - 1) \\
		&= 2 \cdot 3^n.
	\end{align*}
	This proves that our explicit formula is correct for $n+1$, and the proof is complete.
\end{proof}


\subsection{A recurrence relation problem}
\label{sec:recurrence_relation}
Let the sequence $\{ a_n\}$ be defined by the relations $a_0=1$, and 
\[
a_n = 3(a_0 + a_1 + \dots + a_{n-1}) + 1
\]
for $n>0$. 
Compute the first few values of $a_n$, then conjecture an explicit formula for $a_n$, and then prove the formula using strong induction.

Copy and paste the sample solution typed above.  Follow Example 2.5 and Example 2.6 in the book and also the sample solution typed above.

\begin{proof}
	Insert proof
\end{proof}


\section{Divisible by three}
\emph{Use strong induction} to prove that a positive integer is divisible by $3$ if and only if the sum of its digits is divisible by $3$.

Hint: See class notes for a similar statement with $9$.

\begin{proof}
	Insert proof
\end{proof}

\section{Other Questions}
Write your own problem (which can be solved using methods from Chapter 1 and 2) and the solution. 
Ask a classmate to attempt to solve your problem, and walk them through the solution. Write the classmate's name and briefly describe your interaction here. 


Approximately how much time did you spend on this entire assignment?




\end{document}
