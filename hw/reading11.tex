%%%%%%%%%%%%%%%%%%%%%%%%%%%%%%%%%%%%%%%%%%%%%%%%%%%%%%%
% Math 3250 Combinatorics, University of Connecticut
%%%%%%%%%%%%%%%%%%%%%%%%%%%%%%%%%%%%%%%%%%%%%%%%%%%%%%%

% Anything after a percent sign is a comment.

% Necessary first line. \documentclass defines the type of document and some options (for example, try changing the font size (10pt or 11pt or 12pt).
\documentclass[12pt]{amsart}
\usepackage{enumerate} % to be able to enumerate a list using non-numbers
\usepackage{tikz}
%for hypertext references
\usepackage[colorlinks = true,
            linkcolor = blue,
            urlcolor  = blue,
            citecolor = red,
            anchorcolor = green]{hyperref}


\usepackage[letterpaper]{geometry} 
\geometry{tmargin=0.73in,bmargin=-0.0in,lmargin=1.50in,rmargin=1.50in}
\voffset -0.0in

% The part of the tex file between the \documentclass and \begin{document} line is called the preamble.  Things having to do with setting up the document are done in the preamble.  You will not need to mess with it for now, except to change your name and the title of your document. 
% After adding your name next to author, head down to where it says "Start here"

\title{Math3250 Combinatorics Reading HW 11}
%\author{your preferred first and last name:}
%\date{}
\begin{document}

\maketitle

% --------------------------------------------------------
%                         Start here
% --------------------------------------------------------

%\noindent Credit: 
%Write down everyone who helped you, including classmates who contributed to your thought process. Write down Bona's textbook and other written sources you used as well.

\subsection*{Instruction}

\begin{itemize}
\item 
Submit your homework by email (subject line: Math3250 Combinatorics Reading HW 11). 
\item 
You can complete by hand (then scan using your smart phone to produce a PDF file) or type the homework. If you do this by hand, please use a pen or dark pencil so that the scan is readable.
\item
Use B\'ona's ``A Walk through Combinatorics" textbook.
\end{itemize}

\bigskip



\section{Sec 8.2.1 Exponential Generating Functions lecture}
Do one of the following:
\begin{enumerate}[i.]
		\item Watch \href{}{lecture video (12 minutes) of Sec 8.2.1}
	\item Read  \href{https://egunawan.github.io/combinatorics/notes/notes8_2_1recurrence_relations_and_exponential_generating_functions.pdf}{lecture notes for Sec 8.2.1 video}
	\item Read Sec 8.2.1 of the book (pg 180-182)
\end{enumerate}

Write down which option you did. If you watched the video, please specify which platform (Kaltura or YouTube) and what type of device.

\section{Sec 8.2.1 Exponential Gen. Functions: Solve Example 8.19}

Explain in details the solution of Example 8.19 (pg 181). Be as detailed as my solution for Example 8.17.




\section{Watch or read: Sec 8.2.2 Products of Exponential Generating Functions}
Do one of the following:
\begin{enumerate}[i.]
	\item Watch \href{}{lecture video (15 minutes) of Sec 8.2.2}
	\item Read  \href{https://egunawan.github.io/combinatorics/notes/notes8_2_2products_of_exponential_generating_functions.pdf}{lecture notes for Sec 8.2.2 video}
	\item Read Sec 8.2.2 up to the first example.
\end{enumerate}

Write down which option you did. If you watched the video, please specify which platform (Kaltura or YouTube) and what type of device.


\section{Exercise: Sec 8.2.2 Products of Exponential Gen. Functions}
\label{sec:exercise:sec822}

A football coach has $n$ players to work with at today's practice.
First the coach splits the players into two \emph{non-empty} groups, and then the coach puts the members of each group in a line. 
In how many different ways can all this happen? 

(First, attempt to solve this following the same method in solving Example 8.22 explained in the video/ lecture notes. Go to the next page to see a sketch of the solution.)






\section{Survey}
\begin{enumerate}[i.] 
\item Approximately how much time did you spend on this homework?
\item You are encouraged to communicate with your classmates. Write down the resources (for example, B\'ona's textbook or a  \href{https://math.stackexchange.com/}{math.stackexchange.com/} page) you referenced and the people that you talked with.	
\item Questions or comments?
\end{enumerate}


\newpage

\subsection*{Step-by-step outline for solving Problem \ref{sec:exercise:sec822}:}

\begin{itemize}
	\item 
	Let $A(x)$ be the \emph{exponential} generating function (EGF) enumerating the number of ways for (nonzero) players to line up.
	
	\item 
	Explain why $A(x)=\frac{x}{1-x}$.
	
	\item 
	Let $c_n$ be the number of ways for the coach to first splits the players into two non-empty groups then put the players of each group in a line. 
	Let $C(x):= \sum_{n=0}^n c_n \frac{x^n}{n!}$.
	Use the Product formula for exponential generating functions to show that
	$C(x)=\frac{x^2}{(1-x)^2}$.
	
	\item
	Start with the geometric series then differentiate to show that 
	$C(x)=\frac{x^2}{(1-x)^2}=\sum_{n=2}^\infty (n-1)x^n$.
	
	\item  
	Apply the same method as in Example 8.17 and Example 8.22 to conclude that $c_n= n! (n-1)$ for $n \geq 2$.
	
	
\end{itemize}



\end{document}
