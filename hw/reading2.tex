%%%%%%%%%%%%%%%%%%%%%%%%%%%%%%%%%%%%%%%%%%%%%%%%%%%%%%%
% Math 3250 Combinatorics, University of Connecticut
%%%%%%%%%%%%%%%%%%%%%%%%%%%%%%%%%%%%%%%%%%%%%%%%%%%%%%%

% Anything after a percent sign is a comment.

% Necessary first line. \documentclass defines the type of document and some options (for example, try changing the font size (10pt or 11pt or 12pt).
\documentclass[10pt]{amsart}
\usepackage{enumerate} % to be able to enumerate a list using non-numbers
\usepackage{tikz}
%for hypertext references
\usepackage[colorlinks = true,
            linkcolor = blue,
            urlcolor  = blue,
            citecolor = red,
            anchorcolor = green]{hyperref}


\usepackage[letterpaper]{geometry} % from 2016
\geometry{tmargin=0.63in,bmargin=-0.0in,lmargin=1.50in,rmargin=1.50in}
\voffset -0.5in

% The part of the tex file between the \documentclass and \begin{document} line is called the preamble.  Things having to do with setting up the document are done in the preamble.  You will not need to mess with it for now, except to change your name and the title of your document. 
% After adding your name next to author, head down to where it says "Start here"

\title{Math3250 Combinatorics Reading HW 2}
%\author{your preferred first and last name:}
%\date{due at the end of class on d1, Week 2}
\begin{document}

\maketitle

% --------------------------------------------------------
%                         Start here
% --------------------------------------------------------

%\noindent Credit: 
%Write down everyone who helped you, including classmates who contributed to your thought process. Write down Bona's textbook and other written sources you used as well.

\subsection*{Instruction}
Please submit all questions. Both handwritten or typed solutions are accepted.

You are encouraged to discuss the problems with other people in or outside of class. You are also welcome to come see me to show me what you've done so far.

You would need the \href{https://www.worldscientific.com/doi/pdf/10.1142/9789813148857_0001}{first chapter} and second chapter of  B\'ona's ``A Walk Through Combinatorics" textbook (4th, 3rd, or 2nd Ed.)



\bigskip

\section{Topic Requests!}
Are there any interesting topics in Combinatorics that you hope we will cover? If you are not familiar with the content of Combinatorics, you can do a quick internet search and see if any of the information that pops up seems interesting. If you are not sure, then you don't have to request anything.

I apologize ahead of time if you suggest a topic but we don't cover it this semester.

\section{Correction to READING HW 1}
If some of your answers to READING HW 1 problems were incomplete or incorrect, redo your proofs here. First attempt to write your proofs without looking at your class notes.


\section{Write Your Own Pigeon-Hole Principle Problem}\label{sec:write_your_own}
Please write your own problem and solve it using the Pigeon-hole Principle.
(Student's problems may be chosen for future exams' questions.)

\section{Sec 2.1 Weak Induction}
\begin{enumerate}
	\item 
	Copy the algorithm of weak induction (from the first page of Sec 2.1 Weak Induction).
	
	\item Write down the explanation for why the method of induction is valid. Add a few extra phrases and sentences to the book's explanation.
\end{enumerate}


\section{Book Example}
Complete at least one of the following.


\begin{enumerate}
	\item 
	Go to Example 2.1 and attempt to solve it on your own. Then compare your solution with the book's solution.
	
	\item
	Go to Example 2.3 and attempt to solve it on your own. Then compare your solution with the book's solution.
\end{enumerate}

%\section{Quick Check (Don't need to submit)}
%Complete at least one of the following.
%\begin{enumerate}
%	\item 
%	Let $a_0=1$ and let $a_{n+1}=4a_n+1$ for all nonnegative integers $n$. 
%	Prove \emph{using induction} that for all nonnegative integers $n$, the equality 
%	\[
%	a_n = \frac{4^{n+1}-1}{3}
%	\]
%	holds.
%	
%	\item 
%	\emph{Use induction} to prove that if $a_n=1+2+\dots+n$, then 
%	\[
%	a_n = n(n+1)
%	/2
%	\]
%	for all positive integers $n$.
%\end{enumerate}




\section*{Attempt presentation problems (Don't submit)}\label{sec:presentation}
Attempt some of the presentation problems for week 2:\\
\href{https://egunawan.github.io/combinatorics/hw/wk2problems.pdf}{egunawan.github.io/combinatorics/hw/wk2problems.pdf}


Some of the questions will be chosen to be presented during class this week. You may volunteer to present or be an audience member. 

Some of the questions will be picked for future homework and exam questions. 



\section{Survey}
\begin{enumerate}[i.] 
\item Approximately how much time did you spend on this homework?
\item Write down the resources (for example, B\'ona's textbook or a  \href{https://math.stackexchange.com/}{math.stackexchange.com/} page) you referenced and the people that you talked with.	
\item Any questions about this homework?
\end{enumerate}



\end{document}
