%%%%%%%%%%%%%%%%%%%%%%%%%%%%%%%%%%%%%%%%%%%%%%%%%%%%%%%
% Math 3250 Combinatorics, University of Connecticut
%%%%%%%%%%%%%%%%%%%%%%%%%%%%%%%%%%%%%%%%%%%%%%%%%%%%%%%

% Anything after a percent sign is a comment.

% Necessary first line. \documentclass defines the type of document and some options (for example, try changing the font size (10pt or 11pt or 12pt).
\documentclass[12pt]{amsart}
\usepackage{enumerate} % to be able to enumerate a list using non-numbers
\usepackage{tikz}
%for hypertext references
\usepackage[colorlinks = true,
            linkcolor = blue,
            urlcolor  = blue,
            citecolor = red,
            anchorcolor = green]{hyperref}


\usepackage[letterpaper]{geometry} 
\geometry{tmargin=0.9in,bmargin=0.9in,lmargin=1.50in,rmargin=1.50in}
\voffset -0.0in


\begin{document}

\begin{center}\textbf{Math3250 Combinatorics Reading HW 17}\end{center}

% --------------------------------------------------------
%                         Start here
% --------------------------------------------------------



\subsection*{Instruction}


Submit your homework by email (subject: Math3250 Combinatorics Reading HW 17). Use a scanner app to convert to PDF your hand-written work. 

References:
\begin{itemize}
	\item 
\href{https://arxiv.org/abs/math/0311148}{J.~Scott, \emph{Grassmannians and Cluster Algebras}}
\item 
\href{https://math.mit.edu/~apost/papers/tpgrass.pdf}{A.~Postnikov, \emph{Total Positivity, Grassmannians, and Networks}}
\end{itemize}





\section*{Video/Lecture notes}
Either watch the lecture video of plabic graphs (53 minutes) on YouTube, or read the lecture notes \href{https://egunawan.github.io/combinatorics/notes/notes_plabic_graphs.pdf}{lecture notes for plabic graphs}. 

Write down what you did. 
%If you watched the video, please specify (Kaltura/YouTube) and type of device. 
The video is at original speed --- you can play the video at faster speed if you are not taking notes.




\section*{Exercises}

\begin{enumerate}[1.)]
	
\item 
Prove Claim 1 (p. 2 of the lecture notes). If you cannot prove it in general, then please verify Claim 1 for trips starting at $1$, $2$, $3$, $4$, $5$, and $6$ of the plabic graph labeled $D$ (p. 1 of the lecture notes).

\bigskip
	
\item 
Prove the rest of Proposition 2 (p. 3 of the lecture notes): The trip permutation is preserved by a local move (M2) and (M3).

\bigskip
	
\item 
Do the rest of Example 3 (p. 4 of the lecture notes): Compute the face labelings of the plabic graphs $G_1, G_2, G_3, G_4$. 

\bigskip

\item 
Do HW 4 (p. 5 of the lecture notes). Find a natural map from the set of five plabic graphs $G_1, G_2, G_3, G_4, G_5$ (on p. 4) to the set of five triangulations of a pentagon (p. 5).

\end{enumerate}



\section*{Last Section}
\begin{itemize}
	\item 
	Email me with a couple of the above exercises that you would like to show during class on Tue Apr 28.
	\item
	Questions, comments, suggestions?
\end{itemize}



\end{document}
