%%%%%%%%%%%%%%%%%%%%%%%%%%%%%%%%%%%%%%%%%%%%%%%%%%%%%%%
% Math 3250 Combinatorics, University of Connecticut
%%%%%%%%%%%%%%%%%%%%%%%%%%%%%%%%%%%%%%%%%%%%%%%%%%%%%%%

% Anything after a percent sign is a comment.

% Necessary first line. \documentclass defines the type of document and some options (for example, try changing the font size (10pt or 11pt or 12pt).
\documentclass[12pt]{amsart}
\usepackage{enumerate} % to be able to enumerate a list using non-numbers
\usepackage{tikz}
%for hypertext references
\usepackage[colorlinks = true,
            linkcolor = blue,
            urlcolor  = blue,
            citecolor = red,
            anchorcolor = green]{hyperref}


\usepackage[letterpaper]{geometry} 
\geometry{tmargin=0.53in,bmargin=-0.0in,lmargin=1.00in,rmargin=1.00in}
\voffset -0.5in

% The part of the tex file between the \documentclass and \begin{document} line is called the preamble.  Things having to do with setting up the document are done in the preamble.  You will not need to mess with it for now, except to change your name and the title of your document. 
% After adding your name next to author, head down to where it says "Start here"

\title{Math3250 Combinatorics Reading HW 10}
%\author{your preferred first and last name:}
%\date{}
\begin{document}

\maketitle

% --------------------------------------------------------
%                         Start here
% --------------------------------------------------------

%\noindent Credit: 
%Write down everyone who helped you, including classmates who contributed to your thought process. Write down Bona's textbook and other written sources you used as well.

\subsection*{Instruction}
%Please submit all sections. 
Because you will be doing a lot of arithmetic, it might be more convenient to do this homework by hand.
Use B\'ona's ``A Walk through Combinatorics" textbook.

\bigskip


\section*{(Optional) More Fibonacci/Pingala numbers}

(If you are already very comfortable with Sec 8.1.1, you can skip this exercise.)

Try Ch 8 Exercise 5 in the book. Read and then explain the solution given in the textbook (page 196--197).




\section{Combinatorial Proof for Example 8.6}
Write the problem of Example 8.6 (pg 170). Attempt to find a combinatorial proof using bijection (like in Chapter 3).

Read and then explain the solution given in the textbook (Ch 8 Exercise 13, page 200)






\section{Section 8.1.2: The product formula}
Let $f_n$ be the number of ways to pay $n$ dollars using ten-dollar bills, five-dollar bills, and one-dollar bills only. 
Find the ordinary generating function $F(x) = \sum_{n=0}^\infty f_n ~ x^n$. Use it to find a closed form formula for $f_n$. \\


 Hints:
\begin{itemize}
\item Define three sequences, the number of ways to pay $n$ dollars with tens, fives, and ones.
 \item Then use the product formula (Theorem 8.5) twice. 
 \item Follow Example 8.6 and Example 8.8. Use partial fractions to complete the second part of the problem. 
\item 
Compare your answer with the book's solution (Ch 8 Exercise 8, pg 198--199).
\end{itemize}




\section{Sec 8.1.3 Compositions of Generating Functions}

Read page 177. Write down just Def 8.12 and Theorem 8.13.


\section{Example 8.14 Using Theorem 8.13}

Read page 178. 
Write down the problem ins Example 8.14. Explain \emph{just the first paragraph}s of the solution, how Theorem 8.13 is used to solve this problem (you don't need to copy the computation from the book).



\section{Survey}
\begin{enumerate}[i.] 
\item Approximately how much time did you spend on this homework?
%\item Write down the resources (for example, B\'ona's textbook or a  \href{https://math.stackexchange.com/}{math.stackexchange.com/} page) you referenced and the people that you talked with.	
\item Questions or comments?
\end{enumerate}



\end{document}
