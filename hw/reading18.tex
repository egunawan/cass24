%%%%%%%%%%%%%%%%%%%%%%%%%%%%%%%%%%%%%%%%%%%%%%%%%%%%%%%
% Math 3250 Combinatorics, University of Connecticut
%%%%%%%%%%%%%%%%%%%%%%%%%%%%%%%%%%%%%%%%%%%%%%%%%%%%%%%

% Anything after a percent sign is a comment.

% Necessary first line. \documentclass defines the type of document and some options (for example, try changing the font size (10pt or 11pt or 12pt).
\documentclass[10pt]{amsart}
\usepackage{enumerate} % to be able to enumerate a list using non-numbers
\usepackage{tikz}
%for hypertext references
\usepackage[colorlinks = true,
            linkcolor = blue,
            urlcolor  = blue,
            citecolor = red,
            anchorcolor = green]{hyperref}


\usepackage[letterpaper]{geometry} 
\geometry{tmargin=0.9in,bmargin=0.9in,lmargin=1.00in,rmargin=1.00in}
\voffset -0.0in


\begin{document}

\begin{center}\textbf{Math3250 Combinatorics Reading HW 18}\end{center}

% --------------------------------------------------------
%                         Start here
% --------------------------------------------------------



\subsection*{Instruction}


Submit your homework by email (subject: Math3250 Combinatorics Reading HW 18). Use a scanner app to convert to PDF your hand-written work. 

References:
\begin{itemize}
	\item 
\href{https://arxiv.org/abs/math/0311148}{J.~Scott, \emph{Grassmannians and Cluster Algebras}}
\item 
\href{https://math.mit.edu/~apost/papers/tpgrass.pdf}{A.~Postnikov, \emph{Total Positivity, Grassmannians, and Networks}}
\end{itemize}





\section*{Video/Lecture notes}
Either watch the lecture video of plabic graphs Part 2 (16 minutes) on YouTube. 
The lecture notes are in  \href{https://egunawan.github.io/combinatorics/notes/notes_plabic_graphs2.pdf}{lecture notes for plabic graphs part 2}.  It might be faster to watch the video. 

Write down what you did. 
%If you watched the video, please specify (Kaltura/YouTube) and type of device. 
The video is at original speed --- you can play the video at faster speed if you are not taking notes.




\section*{Exercises}

Do \textbf{four} or more of the following exercises

\begin{enumerate}
\setcounter{enumi}{3}
\item 
Attempt HW4' in the new lecture notes.

A triangulation $T_2$ of an $n$-gon is placed ``higher" than $T_1$ 
if $T_2$ is the result of removing a diagonal from $T_1$ and replacing it with another diagonal of higher slope.
This rule works for all $n$ and forms a nice partial order which shows up in many areas of mathematics outside of combinatorics. 

Look at the five plabic graphs from Example 3 (from the previous Reading Homework). Can you come up with a rule that tell you when a plabic graph from Example 3 should be placed ``higher" than another plabic graph? One rule is to simply look at the corresponding triangulation, but try to find a rule that does not rely on triangulations.

\bigskip
	
\item 
Compute the trip permutation of the plabic graph $D_1$. 
How is this trip permutation similar to the trip permutation of $G_1$--$G_5$?

The picture of $D_1$ and partial solution are on p. 6 of note.

\bigskip
	
\item 
Compute the (source) face labeling of $D_1$.

The picture of $D_1$ and partial solution are on p. 5 of note.


\begin{itemize}
\item
What is special about the label of each external face?
\item 
What is special about the label of each internal face?
\end{itemize}

\bigskip

\item \label{itm:seven}
Compute the face labeling of $D_2'$ and compare with the face labeling of $D_1$. 
What changes and what stays the same? 

The picture of $D_2'$ is on p. 6 of note. 


\bigskip 

\item
Follow the steps of applying $(M2')$ and then the square move (as given in question \ref{itm:seven} above) to $D_1$, but for a different square. See picture of $D_1$ on p. 7 of note.




\item 
Compute the trip permutation of the plabic graph $E_1$. 

How is this trip permutation similar to the trip permutation of $D_1$ and the trip permutation of $G_1$--$G_5$?




Picture of $E_1$ and partial solution are on p. 7 of note.



\item 
Compute the (source) face labeling of $E_1$. Hint: each face is labeled by three numbers.

Picture of $E_1$ and partial solution are on p. 7 of note.


\begin{itemize}
	\item
	What is special about the label of each external face?
	\item 
	What is special about the label of each internal face?
\end{itemize}



\end{enumerate}



\section*{Last Section}
\begin{itemize}
	\item 
	Email me with a couple of the above exercises that you would like to show during class on Thurs Apr 30.
	\item
	Questions, comments, suggestions?
\end{itemize}



\end{document}
