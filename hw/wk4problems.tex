%%%%%%%%%%%%%%%%%%%%%%%%%%%%%%%%%%%%%%%%%%%%%%%%%%%%%%%
% Math 3250 Combinatorics, University of Connecticut
%%%%%%%%%%%%%%%%%%%%%%%%%%%%%%%%%%%%%%%%%%%%%%%%%%%%%%%

% Anything after a percent sign is a comment.

% Necessary first line. \documentclass defines the type of document and some options (for example, try changing 12pt to 10pt.
\documentclass[12pt]{amsart}
\usepackage{enumerate} % to be able to enumerate a list using non-numbers
\usepackage{tikz}

\usepackage[letterpaper]{geometry} 
\geometry{tmargin=0.63in,bmargin=-0.0in,lmargin=1.30in,rmargin=1.30in}
\voffset -0.5in

%for hypertext references
\usepackage[colorlinks = true,
            linkcolor = blue,
            urlcolor  = blue,
            citecolor = red,
            anchorcolor = green]{hyperref}


% The part of the tex file between the \documentclass and \begin{document} line is called the preamble.  Things having to do with setting up the document are done in the preamble.  You will not need to mess with it for now, except to change your name and the title of your document. 
% After adding your name next to author, head down to where it says "Start here"

\title{Math3250 Combinatorics Problems Week 4}
%\author{your preferred first and last name}
\date{}
\begin{document}

\maketitle

% --------------------------------------------------------
%                         Start here
% --------------------------------------------------------



\noindent \textbf{
For the sake of learning the tools of Chapter 3, when you work on Problems \ref{sec:functions} - \ref{sec:combinatorics} below, please model your explanations after the the proofs for Thm 3.6, Examples 3.7, 3.11, 3.12 in Sec 3.2; Thm 3.16 in Sec 3.3. }


%\section{6 digits numbers again (Optional)}
%\begin{enumerate}
%\item How many six-digit positive integers are there in which the first and last digits are the same? (For example, 134561)
%\item How many six-digit positive integers start and end with an even digit? (For example, 234560)
%\item How many six-digit positive integers are there in which the first and last digits are of the same parity? (For example, 134561)
%\item How many six-digit positive integers are there in which all the digits are different?
%\end{enumerate}
%For practice, you can follow the style of the proofs for Theorem 3.6 and Example 3.7 in Section 3.2.



\section{Functions}\label{sec:functions}
Let $n$ be a positive integer.
\begin{enumerate}
\item How many injective functions are there from $[n]$ to $[n]$? 

\item How many functions are there from $[n]$ to $[n]$ that are not injective?

\noindent
{\tiny  Hint:
Combine Thm 3.6 (the number of $k$-digit strings over an $n$-element alphabet) and the solution to Example 3.11 (the bijection from the collection of subsets of $[n]$ onto the collection of $n$-digit binary strings.)}
\end{enumerate}


\section{Interns}
A local company has 8 interns from UConn and 12 interns from Eastern Connecticut State University (ECSU). The company would like to form a service committee consisting of the interns. 

\begin{enumerate}
\item How many ways are there to form this committee consisting of 2 UConn interns and 3 ECSU interns? 
\item How many ways are there to form a 5-people committee that contains \emph{at least} one UConn student and one ECSU student?
\end{enumerate}



\section{Polygon diagonals}
Let $n\geq 4$. Consider a convex $n$-gon that is drawn in such a way that no three diagonals intersect in one point. How many intersection points do the diagonals have? (For example, if you draw a pentagon, there are five diagonals and five crossings.) Prove this. 

{\tiny \noindent Hints: 
 Whenever two diagonals cross, it counts as one intersection point. Draw some examples for $n=4,5$. Can you turn this into the problem of choosing $k$-subsets of $[n]$, as in Section 3.3 Choice Problems?}


\section{Rooks}
We would like to place $n$ rooks on an $n\times n$ chess board in such a way that \emph{no rook can attack}. 
In chess, a rook is only allowed to move horizontally or vertically, through any number of unoccupied squares. Please see \url{https://en.wikipedia.org/wiki/Rook_(chess)} for a demonstration. 
In how many ways can we do this? 



\section{Combinatorics class}\label{sec:combinatorics}
A Combinatorics class consists of $n$ freshmen, $n$ sophomores, and $n$ seniors. For example, imagine a class with $12$ students, so $n=4$.

\begin{enumerate}
\item The students are to form $n$ presentation groups of three people each. How many ways can they do this if each group must contain a freshman, a sophomore and a senior?
\item The $n$ seniors are to form a circle. Two circle arrangements are considered identical if each person has the same left neighbor in the circles. How many ways can the $n$ seniors to form a circle? 

{\tiny Hint: 
Imagine breaking the circle so that the seniors now form a straight line.}
 \end{enumerate}



\end{document}
