%%%%%%%%%%%%%%%%%%%%%%%%%%%%%%%%%%%%%%%%%%%%%%%%%%%%%%%
% Math 3250 Combinatorics, University of Connecticut
%%%%%%%%%%%%%%%%%%%%%%%%%%%%%%%%%%%%%%%%%%%%%%%%%%%%%%%

% Anything after a percent sign is a comment.

% Necessary first line. \documentclass defines the type of document and some options (for example, try changing 12pt to 10pt.
\documentclass[12pt]{amsart}


% The part of the tex file between the \documentclass and \begin{document} line is called the preamble.  Things having to do with setting up the document are done in the preamble.  You will not need to mess with it for now, except to change your name and the title of your document. 
% After adding your name next to author, head down to where it says "Start here"

\title{Math3250 Combinatorics Week2 Problem Set}
\author{your preferred first and last name}
\date{}
\begin{document}

\maketitle

% --------------------------------------------------------
%                         Start here
% --------------------------------------------------------

\noindent Credit: Write down everyone who helped you, including classmates who contributed to your thought process (either through sharing insights or through being a sounding board). Write down Bona's textbook and other written sources you used as well.

\subsection*{Instruction}
The exact problems will be announced on Monday during class. 
Please send me an invite via Overleaf by entering my UConn email address.

Note: If you are not sure how to do something, please post on Piazza or come to office hour.

Please remove this instruction section when you are done.

\section{Six distinct integers}
A student wrote six distinct positive integers on the board, and pointed out that none of them had a prime factor larger than $10$.
\begin{enumerate}
    \item 
    Prove that there are two integers on the board that have a common prime divisor.
    \begin{proof}
    Insert proof
    \end{proof}
    \item 
    Could you make the same conclusion if in the first sentence we replaced `six' by `five'? Explain.
    \begin{proof}
    Insert proof
    \end{proof}
\end{enumerate}

\section{Wednesdays}\label{sec:wednesdays} 
The month of January 2019 has five Wednesdays. (Optional: How many months in 2019 contain five Wednesdays?) For any given year, use the Pigeon-hole Principle to determine the possible number of months that contain five Wednesdays.
Click here for hints: %For any given year, there are either four or five months that contain five Wednesdays.\\ To answer this, consider: \\(i) How many Wednesdays are there in one year? \\(ii) Each month has at least 28 days, so each month has at least four Wednesdays.
\begin{proof}
Insert proof
\end{proof}

\section{Soccer Team} 
A soccer team scored a total of 40 goals this season. Nine players scored at least one of those goals. Prove that there are two players among those nine who scored the same number of goals.
\begin{proof}
\end{proof}


\section{Five real numbers whose sum is $100$} 
Let's say I give you a mystery set of five positive real numbers whose sum is $100$. Prove that there are two numbers among them whose difference is at most $10$.

\begin{proof}
\end{proof}

\section{Swimming}\label{sec:swimming}
\begin{enumerate}
    \item \label{itm:June}
    In the month of June, Mr. Consistent went swimming $26$ times, though he never went more than once on the same day. Is it true that there were six consecutive days that he went swimming?
    
    \item Same as part (\ref{itm:June}), but for the month of July instead of June. 
\end{enumerate}

\section{Polynomial degrees}\label{sec:polynomial_degrees}
The product of five given polynomials is a polynomial of degree $21$. Prove that we can choose two of those polynomials so that the degree of their product is at least nine.

Click here for a hint: %Use the general version of the Pigeon-hole principle in Section 1.2.
\begin{proof}
\end{proof}

\section{Faculty members}
A college has $39$ departments, and a total of $262$ faculty members in those departments. Prove that there are three departments in this college that have a total of at least $21$ faculty members.

Click here for a hint: %Use the general version of the Pigeon-hole principle in Section 1.2.
\begin{proof}
\end{proof}


\section{Triples}
Find all triples of positive integers $x < y < z$ for which
\[
\frac{1}{x} + \frac{1}{y} + \frac{1}{z} = 1
\]
holds. 
\begin{proof}
\end{proof}


\section{4-tuples}
Find all quadruples $(a,b,c,d)$ of distinct positive integers so that $a < b < c < d$ and 
\[
\frac{1}{a} + \frac{1}{b} + \frac{1}{c} + \frac{1}{d} = 1. 
\]
 Click here for a hint: %The value $1$ is too small for $a$ but the value $4$ is too big for $a$. If $a=2$, what are the possible values for $b, c, d$? If $a=3$, what are the possible values for $b, c, d$?
\begin{proof}
\end{proof}

\section{Write Your Own Pigeon-Hole Principle Problem}\label{sec:write_your_own}
Please write your own problem and solve it using the Pigeon-hole Principle.
(Student's problems may be chosen for future exams' questions).

\section{Miscellaneous}
\begin{enumerate}
    \item Share your work (at least one problem) and thought process with at least one classmate. Ask them to share their thought process as well. Write down their names and briefly summarize your interactions. A virtual discussion via Piazza or email is fine if you don't have time to interact in person. 
    \item Approximately how much time did you spend on this homework?
\end{enumerate}

\end{document}
