%%%%%%%%%%%%%%%%%%%%%%%%%%%%%%%%%%%%%%%%%%%%%%%%%%%%%%%
% Math 3250 Combinatorics, University of Connecticut
%%%%%%%%%%%%%%%%%%%%%%%%%%%%%%%%%%%%%%%%%%%%%%%%%%%%%%%

% Anything after a percent sign is a comment.

% Necessary first line. \documentclass defines the type of document and some options (for example, try changing 12pt to 10pt.
\documentclass[10pt]{amsart}
\usepackage{enumerate} % to be able to enumerate a list using non-numbers
\usepackage{tikz}
%for hypertext references
\usepackage[colorlinks = true,
            linkcolor = blue,
            urlcolor  = blue,
            citecolor = red,
            anchorcolor = green]{hyperref}


% The part of the tex file between the \documentclass and \begin{document} line is called the preamble.  Things having to do with setting up the document are done in the preamble.  You will not need to mess with it for now, except to change your name and the title of your document. 
% After adding your name next to author, head down to where it says "Start here"

\title{Math3250 Combinatorics Week8 Problem Set}
\author{your preferred first and last name}
\date{}
\begin{document}

\maketitle

% --------------------------------------------------------
%                         Start here
% --------------------------------------------------------

\noindent Credit: 
Write down everyone who helped you, including classmates who contributed to your thought process. Write down Bona's textbook and other written sources you used as well.

\subsection*{Instruction}
Please send me an invite via Overleaf. 
You are encouraged to work with other people, but this week your must write your own solution.

\bigskip 


\section{Section 8.1.1: Recurrence relations and generating functions}
Type the final answers in \LaTeX below. 
Because the equations may be tedious to type, you can submit handwritten work at the beginning of class (or type your work below, if you prefer). 
If you have to take partial fractions, 
you may use a computing tool (or not), but do the rest of your work by hand. 

\begin{enumerate}
\item 
Let $a_0=1$ and $a_{n+1} = 3 a_n + 2^n$ if $n \geq 0$.
\begin{itemize}
\item Find an explicit formula for the ordinary generating function of the sequence $\{ a_n\}_{n \geq 0}$.
\item Use the previous part to compute an explicit formula for $a_n$ for $a_n \geq 0$.
\end{itemize} 

\item 
Let $a_0=1, a_1=4$ and $a_{n+2} = 8 a_{n+1} - 16 a_n$ if $n \geq 0$.
\begin{itemize}
\item Use the recurrence relation to find an explicit formula for the ordinary generating function 
$A(x)=\sum_{n=0}^\infty a_n x^n$. 
\item Use the previous part to compute an explicit formula for $a_n$ for $n\geq 2$.
\end{itemize} 

\noindent Uncomment for a hint:
%The approach followed in the solution of Example 8.3 (pg 168) will work. 

\item 
A mice population multiplies so that at the end of each year, its size is the double of its size a year before, plus $1000$ more mice.
Assuming that originally we released $50$ mice, how many of them will we have at the end of the $n$th year?
Use (ordinary) generating functions.
\end{enumerate}

\begin{proof}[Answer]
\begin{enumerate}
\item 
An explicit formula for the ordinary generating function is \[\boxed{insert ~ formula ~ in ~ terms ~ of ~ x}.\]

An explicit formula for $a_n$ is $\boxed{insert ~ formula ~ in ~ terms ~ of ~ n}$.

\item 
An explicit formula for the ordinary generating function is \[\boxed{insert ~ formula ~ in ~ terms ~ of ~ x}.\]

An explicit formula for $a_n$ is $\boxed{insert ~ formula ~ in ~ terms ~ of ~ n}$.


\item 
We will have $\boxed{insert ~ formula ~ in ~ terms ~ of ~ n}$ mice at the end of the $n$th year.

\end{enumerate}
\end{proof}




\section{Section 8.1.2: The product formula}
Let $f_n$ be the number of ways to pay $n$ dollars using ten-dollar bills, five-dollar bills, and one-dollar bills only. 
Find the ordinary generating function $F(x) = \sum_{n=0}^\infty f_n ~ x^n$. Use it to find a closed form formula for $f_n$. 

\noindent Uncomment for a hint:
%Define three sequences, the number of ways to pay $n$ dollars with tens, fives, and ones, then use the product formula (Theorem 8.5) twice. Follow Example 8.6 and Example 8.8. Use partial fractions to complete the second part of the problem. 




\section{All integer partitions}
For $n \geq 0$, let $p(n)$ denote the number of partitions of the integer $n$.
Prove that
\[
\sum_{n=0}^\infty p(n) \, x^n = \prod_{k=1}^\infty \frac{1}{1-x^k}
\]

\noindent Hint: Your proof should be almost the same as the proof of Example 8.9 (pg 173). Since this class is an introductory combinatorics class, your proof should be at least twice as detailed as Bona's. 



\section{No part divisible by three}
Show that the number of partitions of $n$ for which no part appears more than twice is equal to the number of partitions of $n$ for which no part is divisible by $3$. For instance, when $n = 5$ there are five partitions of the first type 
\begin{itemize}
\item $(5)$, 
\item $(4,1)$, 
\item $(3,2)$, 
\item $(3,1,1)$, 
\item $(2,2,1)$ 
\end{itemize}
and five of the second type 
\begin{itemize}
\item $(5)$, 
\item $(4,1)$,
\item $(2,2,1)$,
\item $(2,1,1,1)$, 
\item $(1,1,1,1,1)$. 
\end{itemize}
Use generating functions.

\noindent Uncomment for hint 1:
%We proved a similar problem on Wednesday and Friday, week 7, showing that the generating function for the number of partitions of $n$ for which no part appears more than \emph{once} is equal to the generating function for the number of partitions of $n$ for which no part is divisible by $2$. This generating function is given in Sec 8.1.2 of Bona (Example 8.11) without proof. You can model your proof after mine.  \\
\noindent Uncomment for a hint 2:
%In Example 8.11, we took advantage of the fact that $(1+x^i)(1-x^i) = 1-x^{2i}$ to show that the two generating functions are equal. For this problem, you should keep $(1-x^i)$ the same, but replace $1-x^{2i}$ with $1-x^{3i}$ and replace $(1+x^i)$ with the appropriate factor. 




\section{Combinatorial proof}
Copy and paste here any \textbf{one} of the problems (or subproblems) given above, and find a combinatorial proof for the result.\footnote{If you don't come up with a combinatorial proof by the deadline, you can instead write a proof using any method (without generating functions), for example, induction. }







\section{Polygon diagonals (from week 4 problem set)}
Let $n\geq 4$. Consider a convex $n$-gon that is drawn in such a way that no three diagonals intersect in one point. How many intersection points do the diagonals have? (For example, if you draw a pentagon, there are five diagonals and five crossings.) Prove this. 

\noindent Uncomment for a hint:
%Whenever two diagonals cross, it counts as one intersection point. Draw some examples for $n=4,5$. Can you turn this into the problem of choosing $k$-subsets of $[n]$, as in Section 3.3 Choice Problems? 








\section{Miscellaneous}
\begin{enumerate}[i.]
    \item Share your work (at least one problem) and thought process with at least one classmate. Ask them to share their thought process as well. Write down their names and briefly summarize your interactions. A virtual discussion via Piazza or email is fine if you don't have time to interact in person. 
    \item Approximately how much time did you spend on this homework?
\end{enumerate}

\end{document}
