%%%%%%%%%%%%%%%%%%%%%%%%%%%%%%%%%%%%%%%%%%%%%%%%%%%%%%%
% Math 3250 Combinatorics, University of Connecticut
%%%%%%%%%%%%%%%%%%%%%%%%%%%%%%%%%%%%%%%%%%%%%%%%%%%%%%%

% Anything after a percent sign is a comment.

% Necessary first line. \documentclass defines the type of document and some options (for example, try changing the font size (10pt or 11pt or 12pt).
\documentclass[10pt]{amsart}
\usepackage{enumerate} % to be able to enumerate a list using non-numbers
\usepackage{tikz}
%for hypertext references
\usepackage[colorlinks = true,
            linkcolor = blue,
            urlcolor  = blue,
            citecolor = red,
            anchorcolor = green]{hyperref}


\usepackage[letterpaper]{geometry} 
\geometry{tmargin=0.03in,bmargin=0.0in,lmargin=0.50in,rmargin=0.50in}
\voffset -0.3in

% The part of the tex file between the \documentclass and \begin{document} line is called the preamble.  Things having to do with setting up the document are done in the preamble.  You will not need to mess with it for now, except to change your name and the title of your document. 
% After adding your name next to author, head down to where it says "Start here"

\title{Math3250 Combinatorics Reading HW 12}
%\author{your preferred first and last name:}
%\date{}
\begin{document}

\maketitle

% --------------------------------------------------------
%                         Start here
% --------------------------------------------------------

%\noindent Credit: 
%Write down everyone who helped you, including classmates who contributed to your thought process. Write down Bona's textbook and other written sources you used as well.

\subsection*{Instruction}

\begin{itemize}
\item 
Submit your homework by email (subject line: Math3250 Combinatorics Reading HW 12). 
\item Your answers don't require math symbols beyond letters and numbers, so you can type your answers in the body of an email or in a word editor.
\item 
You can also complete by hand (then scan using your smart phone to produce a PDF file) or in \LaTeX. %If you do this by hand, please use a pen or dark pencil so that the scan is readable.
\item
Ref: textbook \emph{Combinatorics and Graph Theory} by Harris, Hirst, and Mossinghoff (HHM) Sec 1.1.1--1.1.2 and
 B\'ona's ``A Walk through Combinatorics" textbook, Chapter 9
\end{itemize}





\section{Good Will Hunting}


Watch \href{https://www.youtube.com/watch?v=iW_LkYiuTKE}{The problem in Good Will Hunting - Numberphile} (5 mins). Answer:
What problem is explained in the video?





\section{Watch or read: Sec 1.1.1--1.1.2, p. 1--8 of HMM}
Get \emph{Combinatorics \& Graph Theory} by Harris, Hirst, \& Mossinghoff (HHM) from HuskyCT: Graph Theory textbook.

Do one of the following:
\begin{enumerate}[i.]
	\item Watch \href{}{lecture video of Sec 1.1.1 and 1.1.2 (part A and B)}
	\item Read only the parts highlighted in color  \href{https://egunawan.github.io/combinatorics/notes/notes1_1_2basics.pdf}{lecture notes for Sec 1.1.1 and 1.1.2 video}
	\item Read Sec 1.1.1--1.1.2 of HHM, p. 1--8, but focus on the vocab words highlighted in the videos/lecture notes.
\end{enumerate}

Write down which option you did. If you watched the video, please specify (Kaltura or YouTube) and what type of device.



\section{HHM Sec 1.1.2 Exercises}

Attempt two or more of the following exercises from the textbook \emph{Combinatorics and Graph Theory} by Harris, Hirst, and Mossinghoff (HHM) from HuskyCT.

\subsection*{
HHM Sec 1.1.2, p.9: Exercise 1
}
(maximum number of edges in a graph on $n$ vertices)

\subsection*{
HHM Sec 1.1.2, p.9: Exercise 2
}
(degree sequence having at least a pair of repeated entries)

\subsection*{
HHM Sec 1.1.2, p.9: Exercise 3a,b
}
(how many paths, i.e. walks with no repeated vertices?)

\subsection*{
HHM Sec 1.1.2, p.9: Exercise 3c,d
}
(how many circuits, i.e. closed trails, distinct edges?)


\section{B\'ona Ch 9 Exercises}

Complete three or more of the following exercises from B\'ona's textbook.
There are brief solutions to the exercises at the back of the chapter, so you can at least copy the book's solutions.
You should include more details in your proofs.

\subsection*{
Bona Ch 9, p.219: Exercise 1} 
(make all streets one-way so that you can never return to a point you have left) Go to the end of the chapter for a brief solution.

\subsection*{
Bona Ch 9, p.219: Exercise 3
}
(about the simple graph on 10 vertices) Go to the end of the chapter for a brief solution.

\subsection*{
Bona Ch 9, p.219: Exercise 4
}
(about the simple graph on 9 vertices) Go to the end of the chapter for a brief solution.

\subsection*{
Bona Ch 9, p.220: Exercise 7
}
(the number of possible simple graphs with $n$ vertices) Go to the end of the chapter for a brief solution.

\subsection*{
Bona Ch 9, p.220: Exercise 10
}
(the number of people having an odd number of siblings is even) Apply the ``First Theorem of Graph Theory" of HHM. Go to the end of the chapter for a brief solution.



\section{Presentations}
Pick any two of the nine exercises listed above, and prepare to explain your attempts during class meeting on Tuesday, March 31. 

It's fine (and maybe better for you) if you pick questions that you don't feel 100\% about.

You are welcome to send me your solution before class meeting so that I can set it up ahead of time (so that you only need to speak and not worry about your camera setting).






\section{Miscellaneous}
\begin{enumerate}[i.]  
\item Approximately how much time did you spend on this homework (including reading or watching the videos)?
\item You are encouraged to communicate with your classmates. Write down the resources (for example, a textbook or a  \href{https://math.stackexchange.com/}{math.stackexchange.com/} page) you referenced and the people that you talked with.	
\item 
What can I do to improve your remote learning experience? 
Questions or other comments?
\end{enumerate}




\end{document}
