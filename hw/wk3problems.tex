%%%%%%%%%%%%%%%%%%%%%%%%%%%%%%%%%%%%%%%%%%%%%%%%%%%%%%%
% Math 3250 Combinatorics, University of Connecticut
%%%%%%%%%%%%%%%%%%%%%%%%%%%%%%%%%%%%%%%%%%%%%%%%%%%%%%%

% Anything after a percent sign is a comment.

% Necessary first line. \documentclass defines the type of document and some options (for example, try changing 12pt to 10pt.
\documentclass[12pt]{amsart}


\usepackage[letterpaper]{geometry} 
\geometry{tmargin=0.63in,bmargin=-0.0in,lmargin=1.50in,rmargin=1.50in}
\voffset -0.5in

\usepackage{enumerate} % to be able to enumerate a list using non-numbers
\usepackage{tikz}


% The part of the tex file between the \documentclass and \begin{document} line is called the preamble.  Things having to do with setting up the document are done in the preamble.  You will not need to mess with it for now, except to change your name and the title of your document. 
% After adding your name next to author, head down to where it says "Start here"

\title{Math3250 Combinatorics Problems Week 3}
%\author{your preferred first and last name}
\date{}
\begin{document}

\maketitle

% --------------------------------------------------------
%                         Start here
% --------------------------------------------------------




\section{Divisible by eleven}
Prove that a positive integer with digits $a_1, a_2, \dots, a_n$ is divisible by $11$ if and only if $a_1 - a_2 + a_3 - \dots + (-1)^{n-1} a_n$ is divisible by $11$.

%Hint: Use Section 2.2 Strong Induction. The initial step is to verify the statement for the integers $1$, $2$, $\dots$, $98$, $99$ with fewer than $3$ digits.
\begin{proof}
Insert proof
\end{proof}



\section{6 digit numbers}
\begin{enumerate}[a)]
    \item  How many $6$ digit numbers are there (leading zeros, e.g. $001223$ not allowed)? 
    \item How many of these are even? 
    \item How many 6 digit numbers are there with exactly one 7? 
    \item How many 6 digit numbers are there that are the same forward and backwards (e.g., $890098$)?
\end{enumerate}

\begin{proof}
Insert answers and explanations
\end{proof}



\section{BAA, ABA, AAB}
How many 3 digit positive integers contain two (but not three) digits? Example of such digits are $122$, $343$, $660$.


\section{Sandwich shop} 
A sandwich shop has $4$ protein options. 
It also has $6$ veggies: lettuce, sprouts, carrots, onion, tomato and pickles. It carries $5$ sauces: mustard, catsup, mayo, sirachi, and vinegar. How many sandwiches can be made from one protein, one veggie and AT MOST one sauce?
\begin{proof}
Insert answer and a brief reasoning.
\end{proof}



\section{Connecticut}
Compute the number of ways to create a  list (of size $11$) of the letters of the word CONNECTICUT. 
\begin{proof}
Insert answer and a brief reasoning.
\end{proof}

\section{Alternating parity}
\begin{enumerate}[a)]
    \item Warm-up: In how many ways can the elements of $[3]$ be permuted so that the sum of every two consecutive elements in the permutation is odd? 
    In how many ways can the elements of $[4]$ be permuted so that the sum of every two consecutive elements in the permutation is odd?

    \item (Optional) Compute this for $[5]$ as well.

    \item In how many ways can the elements of $[n]$ be permuted so that the sum of every two consecutive elements in the permutation is odd?
\end{enumerate}



\section{Write Your Own Problem}\label{sec:write_your_own}
Please write your own problem and solve it using theorems or concepts from Section 3.1 of Bona.
(Student's problems may be chosen for future exams' questions). Test the difficulty level of your problem by sharing it with another math student. 

\end{document}
