%%%%%%%%%%%%%%%%%%%%%%%%%%%%%%%%%%%%%%%%%%%%%%%%%%%%%%%
% Math 3250 Combinatorics, University of Connecticut
%%%%%%%%%%%%%%%%%%%%%%%%%%%%%%%%%%%%%%%%%%%%%%%%%%%%%%%

% Anything after a percent sign is a comment.

% Necessary first line. \documentclass defines the type of document and some options (for example, try changing 10pt to 11pt or 12pt.
\documentclass[10pt]{amsart}
\usepackage{enumerate} % to be able to enumerate a list using non-numbers
\usepackage{tikz}
%for hypertext references
\usepackage[colorlinks = true,
            linkcolor = blue,
            urlcolor  = blue,
            citecolor = red,
            anchorcolor = green]{hyperref}


\usepackage[letterpaper]{geometry} % from 2016
\geometry{tmargin=0.93in,bmargin=-0.0in,lmargin=1.50in,rmargin=1.50in}
\voffset -0.5in

% The part of the tex file between the \documentclass and \begin{document} line is called the preamble.  Things having to do with setting up the document are done in the preamble.  You will not need to mess with it for now, except to change your name and the title of your document. 
% After adding your name next to author, head down to where it says "Start here"

\title{Math3250 Combinatorics Reading HW 1}
%\author{your preferred first and last name:}
%\date{due at the end of class on d2, Week 1}
\begin{document}

\maketitle

% --------------------------------------------------------
%                         Start here
% --------------------------------------------------------

%\noindent Credit: 
%Write down everyone who helped you, including classmates who contributed to your thought process. Write down Bona's textbook and other written sources you used as well.

\subsection*{Instruction}
The following problems will be presented during class on Thursday, Jan 23. 
If you have attempted the problem before class (even if you don't have complete solution), you can volunteer to present the problems in a group of size 1 or 2. 
You can also volunteer to earn audience participation that day instead. 


Problems \ref{sec:eight_integers}, \ref{sec:six_positive_integers}, and \ref{sec:polynomial_degrees} will be graded based on effort.

Please use the Pigeon-Hole Principle (PHP) to solve all problems. 
Use the first chapter of Bona's textbook:
\begin{footnotesize}
\href{https://www.worldscientific.com/doi/pdf/10.1142/9789813148857_0001}{www.worldscientific.com/doi/pdf/10.1142/9789813148857\_0001}
\end{footnotesize}



\bigskip




\section{Example 1.3 from Bona}
A chess tournament has $n$ participants, and any two players play exactly one game against each other. Is it true that in any given point of time, there are two players who have finished the same number of games?

\begin{enumerate}
\item 
First, attempt the following problem on your own or with other students (without looking up the textbook's solution)

\item Read the solution given under Example 1.3. 

\item What play the role of ``boxes" in this problem?

\item What play the role of ``balls" in this problem?

\item Use the book's solution to write down a solution in your own words. It should be wordier than the book's solution. Try to first write it without looking at the book's solution.
\end{enumerate}




\section{Eight integers}\label{sec:eight_integers}
Given eight distinct integers, prove that there are two integers whose difference is divisible by seven.

\begin{enumerate}
	\item What play the role of ``boxes" in this problem?
	
	\item What play the role of ``balls" in this problem?
\end{enumerate}



\section{Six positive integers}\label{sec:six_positive_integers}
A student wrote six distinct positive integers on the board, and pointed out that none of them had a prime factor larger than $10$. 






\begin{enumerate}
	\item Prove that there are two integers on the board that have a common prime divisor.
	
	\item Could you make the same conclusion if in the first sentence we replaced ``six" by ``five"?
	
	
	\item What play the role of ``boxes" in this problem?
	
	\item What play the role of ``balls" in this problem?
\end{enumerate}



\section{Polynomial degrees}\label{sec:polynomial_degrees}
The product of five given polynomials is a polynomial of degree $21$. Prove that we can choose two of those polynomials so that the degree of their product is at least nine.







\section{Survey}
\begin{enumerate}[i.] 
\item Approximately how much time did you spend on this homework?
\item Any questions so far?
\end{enumerate}

\end{document}
